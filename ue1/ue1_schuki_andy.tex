\documentclass[DIN, pagenumber=false, fontsize=11pt, parskip=half]{scrartcl}

\usepackage{ngerman}
\usepackage[utf8]{inputenc}
\usepackage[T1]{fontenc}
\usepackage{textcomp}
\usepackage{graphicx}
\usepackage{amsmath}
\usepackage{ amssymb }
\usepackage{pgfplots}

% for matlab code
% bw = blackwhite - optimized for print, otherwise source is colored
%\usepackage[framed,numbered,bw]{mcode}

% for other code
%\usepackage{listings}

\setlength{\parindent}{0em}

% set section in CM
\setkomafont{section}{\normalfont\bfseries\Large}

\newcommand{\mytitle}[1]{{\noindent\textbf{#1}}}
\newcommand{\sol}{\underline{Solution:} }
\usepackage{fancyhdr}

\pagestyle{fancy}
\fancyhf{}
\rhead{ Andreas Rist \\  Joshua Marben}
\lhead{University Tübingen \\Self Driving Cars WiSe 2018/19}
\fancyfoot{}
\lfoot{{ andreas.rist@student.uni-tuebingen.de}\\ { joshua.marben@student.uni-tuebingen.de}}
\rfoot{ Page \thepage}
\renewcommand{\headrulewidth}{2pt}
\renewcommand{\footrulewidth}{1pt}
\renewcommand\thesection{\underline {Task \arabic{section}:}}
%===================================
\begin{document}
\mytitle{{\huge Excercise 1} \hfill \today}
%===================================
\section{Network Design }
\begin{enumerate}
	\item[b)]The module $training.py$ contains the training loop for the network. Read and understand its
	function $train$. Why is it necessary to divide the data into batches? What is an epoch? What do
	lines $43$ to $48$ do? Please answer shortly and precisely.\\
	\sol It is necessary to divide your data into batches for 2 reasons. First there is less memory used to train the network and second the network learns faster with smaller batch sizes. A lot papers propagate a batch size of 64 as Optimal learning.\\
	An epoch is is one forward and one backward pass of all training data.\\
	We take the batch of our training data, calculate the prediction of the network, compare them to the target values using the cross-entropy-loss-function. Then the loss is used to propagate the error back to the network to adapt the weights in the network. 
	\item[c)] Define the set of action-classes you want to use and complete the class-methods actions to classes
	and scores to action in $network.py$\\
	\sol The provided data of the expert imitations comes as a set of free:
	\[(steer,gas,brake)\]
	with
	\begin{align*}
		steer&\in\{-1,0,1\}\\
		gas&\in \{0,0.5\}\\
		break&\in \{0,0.8\}
	\end{align*}
	We want to use following classes:\\ 
		\{steer\_left\}\\
		\{steer\_right\}\\
		\{steer\_left\ and brake\}\\
		\{steer\_right and brake\}\\		
		\{brake\}\\
		\{gas\}\\
		\{\}\\
		Accelerating and steering makes no sense to us, because it would conclude in a donut. 
\end{enumerate}

\end{document}
